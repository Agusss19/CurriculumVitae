%% start of file `template.tex'.
%% Copyright 2006-2013 Xavier Danaux (xdanaux@gmail.com).
%
% This work may be distributed and/or modified under the
% conditions of the LaTeX Project Public License version 1.3c,
% available at http://www.latex-project.org/lppl/.


\documentclass[11pt,a4paper,sans]{moderncv}        % possible options include font size ('10pt', '11pt' and '12pt'), paper size ('a4paper', 'letterpaper', 'a5paper', 'legalpaper', 'executivepaper' and 'landscape') and font family ('sans' and 'roman')

% moderncv themes
\moderncvstyle{classic}                             % style options are 'casual' (default), 'classic', 'oldstyle' and 'banking'
\moderncvcolor{blue}                               % color options 'blue' (default), 'orange', 'green', 'red', 'purple', 'grey' and 'black'
%\renewcommand{\familydefault}{\sfdefault}         % to set the default font; use '\sfdefault' for the default sans serif font, '\rmdefault' for the default roman one, or any tex font name
%\nopagenumbers{}                                  % uncomment to suppress automatic page numbering for CVs longer than one page

% character encoding
\usepackage[utf8]{inputenc}                       % if you are not using xelatex ou lualatex, replace by the encoding you are using
%\usepackage{CJKutf8}                              % if you need to use CJK to typeset your resume in Chinese, Japanese or Korean

% adjust the page margins
\usepackage[scale=0.75]{geometry}
%\setlength{\hintscolumnwidth}{3cm}                % if you want to change the width of the column with the dates
%\setlength{\makecvtitlenamewidth}{10cm}           % for the 'classic' style, if you want to force the width allocated to your name and avoid line breaks. be careful though, the length is normally calculated to avoid any overlap with your personal info; use this at your own typographical risks...

% personal data
\name{Agustin}{Garcia}
\title{Ingeniero en Computación}                               % optional, remove / comment the line if not wanted
\address{}{Nacimiento: 17/06/1995}{Montevideo, Uruguay}% optional, remove / comment the line if not wanted; the "postcode city" and and "country" arguments can be omitted or provided empty
\phone[mobile]{(+598)~99~785~584}                   % optional, remove / comment the line if not wanted
%\phone[fixed]{(011)~4219~2261}                    % optional, remove / comment the line if not wanted
%\phone[fax]{+3~(456)~789~012}                      % optional, remove / comment the line if not wanted
\email{agusss.garcia19@gmail.com}                               % optional, remove / comment the line if not wanted
\homepage{www.github.com/agusss19}                         % optional, remove / comment the
%line if not wanted
               % optional, remove / comment the line if not wanted
\photo[64pt][0.4pt]{fotoCarnet}                       % optional, remove / comment the line if not wanted; '64pt' is the height the picture must be resized to, 0.4pt is the thickness of the frame around it (put it to 0pt for no frame) and 'picture' is the name of the picture file
\quote{Presentación}  % optional, remove / comment the line if not wanted

% to show numerical labels in the bibliography (default is to show no labels); only useful if you make citations in your resume
%\makeatletter
%\renewcommand*{\bibliographyitemlabel}{\@biblabel{\arabic{enumiv}}}
%\makeatother
%\renewcommand*{\bibliographyitemlabel}{[\arabic{enumiv}]}% CONSIDER REPLACING THE ABOVE BY THIS

% bibliography with mutiple entries
%\usepackage{multibib}
%\newcites{book,misc}{{Books},{Others}}
%----------------------------------------------------------------------------------
%            content
%----------------------------------------------------------------------------------
\begin{document}
%\begin{CJK*}{UTF8}{gbsn}                          % to typeset your resume in Chinese using CJK
%-----       resume       ---------------------------------------------------------
\makecvtitle
Ingeniero en computación proactivo, automotivado con ánimos de crecer profesionalmente. 
Dispuesto a afrontar cualquier desafío ingenieril que se presente, brindando una solución creativa y eficiente.
Amplios conocimientos en Ingeniería de Software, metodologías del desarrollo ágil, Matemáticas y Electrónica, adquiridas por medio de la Facultad y la experiencia en la carrera profesional.
Dispuesto al aprendizaje continuo y al trabajo tanto grupal como individual.

\section{Educación}
\cventry{2013--2017}{Ingeniería en Computación}{Universidad Nacional de la Plata (UNLP)}{La Plata}{\textit{Carrera Universitaria desarrollada en conjunto con la Facultad de Informática e Ingeniería UNLP y completada con promedio 8,88. Reconocido como uno de los mejores promedios del ciclo lectivo 2017-2018. Entrega del premio Joaquín V. Gonzalez}}{}

\cventry{2020--2020}{React, Redux}{}{}{\textit{Curso online realizado con el fin de obtener las herramientas esenciales de las librerías de React y Redux }{\href{https://www.linkedin.com/posts/agusss19_redux-certificate-of-graduation-activity-6687055289507692544-I4oA}{Link del certificado en Linkedin}}}{}

\cventry{2018--2019}{.NET}{}{}{\textit{Estudio realizado para la obtención de los conceptos básicos y avanzados relacionados con el framwork .NET MVC 5 y .NET Core en C\#}}{}

\cventry{2018--2018}{Angular}{}{}{\textit{Curso online realizado con el objetivo de conocer las herramientas que otorga el framework a fin de satisfacer distintas necesidades laborales.}}{}

\cventry{2018--2018}{Blockchain}{}{}{\textit{Mini curso realizado para poder desarrollar un programa con esta tecnología de punta. El sistema está hecho en el lenguaje Golang.}}{}

\cventry{2017--2018}{OpenGL}{}{}{\textit{Estudio realizado con el fin de adquirir las herramientas básicas y avanzadas para la utilización de la tecnología OpenGL tanto en Java, como en C++}}{}

\cventry{2006-2010}{Ingles}{Oxford-Instituto de Ingles}{La Plata}{\textit{Estudio realizado para la comprensión del lenguaje.}}{}

\cventry{2007--2012}{Estudio secundario}{Colegio Nacional Rafaél Hernández}{La Plata}{\textit{Estudios secundarios completos, con orientación a las ciencias Informáticas}}{}


\section{Experiencia}
\cventry{2020--actualidad}{Desarrollador Fullstack}{(.NET Core, React, Redux, NextJs) \href{https://uruit.com/}{UruIT}}{Montevideo}{}{Desarrollo de un sistema para la empresa SupremeGolf en Estados Unidos con el fin de vender horas para jugar al golf. El sistema fue construido usando React (Hooks), Redux y NextJs para el Frontend, .NET Core (C\#) para el Backend y el ORM Dapper para la comunicación con la base de datos de SQLServer. La aplicación está orientada a microservicios (CQRS), utiliza las últimas técnologías cloud de Azure y paradigmas tales como DDD e inyección de dependencias. Algunos de los componentes utilizados en la solución son AppServices, Queues, Topics, Funciones y Redis. Para el testing de la aplicación se utilizó Jest en el Frontend y XUnit en el Backend. El deploy a los ambientes se realizó utilizando el servicio de Azure DevOps a fin de compilar la solución, ejecutar los unit tests, los de integración y finalmente empaquetar y copiar los archivos resultantes al servidor destino. Se requirió, además, una participación activa en el análisis de los requerimientos del cliente y en la definición de las correspondientes tareas. Incluso se le brindó asesoramiento técnico ante algunas necesidades de negocio particulares. La metodología de desarrollo era Scrum, incluyendo todas las instancias con el cliente: daily, grooming, planning, retrospective.}

\cventry{2019--2019}{Desarrollador Fullstack}{(.NET Core, React, Redux) \href{https://www.novit.com.ar/}{Novit}}{CABA}{}{Desarrollo de un sistema para la empresa CityParking en Colombia de manera remota desde La Plata. El objetivo de negocio del proyecto consistía en el manejo de los cupos de estacionamientos para autos, motos y bicicletas de todo el país. El sistema fue construido utilizando React y Redux para el frontend, en conjunto con la tecnología de PWA para que simulara una aplicación móvil y pudiera usarse incluso cuando no existiera una conexión a internet. Para el backend se requirió de conocimientos de .NET Core (C\#) y el ORM de EntityFramework para la base de datos de SqlServer. Inicialmente la arquitectura era en capas y posteriormente se migró a slices (CQRS y DDD). Para el testing de la aplicación se utilizó Jest en el Frontend y XUnit en el Backend. El deploy a los ambientes se realzió utilizando contendores Docker en Azure. La metodología de desarrollo a utilizar fue Scrum con las herramientas de Trello (y PowerUps). Se requirió, además, una participación activa en el análisis de los requerimientos del cliente y en la definición de las correspondientes tareas. Incluso se le brindó asesoramiento técnico ante algunas necesidades de negocio particulares.}

\cventry{2017--2019}{Desarrollador Fullstack}{(.NET Core, React, Redux) \href{https://www.cgp.gba.gov.ar/}{Contaduría General de la Provincia de Buenos Aires}}{La Plata}{}{Creación de aplicaciones internas, con el fin de automatizar actividades y procesos que antiguamente se realizaban de manera manual. A cada aplicación se le dio una solución a la medida del problema a resolver en cuestión. Los sistemas fueron construidos usando React y Redux para el Frontend, .NET Core (C\#) para el Backend y el ORM de NHibernate para la comunicación con las bases de datos de Oracle y SQLServer. 
Para el desarrollo de estos sistemas, se requirió amplios conocimientos en arquitectura de software (capas, vertical slices, CQRS, DDD, inyección de dependencias). Para el testing de las mismas se utilizó Jest en el Frontend y XUnit en el Backend. El deploy de los distintos ambientes se realizó utilizando la herramienta de Jenkins. Fue necesaria la creación de un script en Powershell capaz de crear el sitio en el IIS, compilar la solución, ejecutar los unit tests, los de integración y finalmente empaquetar y copiar los archivos resultantes al servidor destino (Windows Server). De esta manera se llevo a cabo el paradigma de integración/deploy continuo. La metodología de desarrollo fue Scrum utilizando herramientas como Trello, Slack.}

\cventry{2018--2019}{Profesor diplomado}{(Java) \href{https://www.info.unlp.edu.ar/}{Universidad Nacional de La Plata}}{La Plata}{}{Ayudante diplomado de las materias Programación 3 (Java) y Taller de Lenguajes 2 (Java) dictadas en la Facultad de Informática de la UNLP. Ambas materias consisten en el aprendizaje del lenguaje en cuestión, en conjunto con los patrones de diseño más comunes que se utilizan en el mundo empresarial.}

\cventry{2017--2018}{Desarrollador de Software}{(Java, C++) \href{https://www.linti.unlp.edu.ar/}{Laboratorio de Investigación en Nuevas Tecnologías Informáticas}}{La Plata}{}{Desarrollo del sistema de investigación “Creación de dispositivos basados en sensores y software específico para mejorar el entrenamiento de deportistas”. El proyecto fue aprobado en la convocatoria a “Proyectos de Desarrollo de Aplicaciones e Innovación” de la Facultad de Informática del año 2017. El sistema consiste en un componente que controla los microcontroladores, denominado "Terminal" y una serie de dispositivos periféricos denominados "Tortugas". El primero corre en móviles Android y fue escrito usando Java. Los periféricos, por otro lado fueron programados usando C++ y el SDK del ESP8266. El objetivo era sensar la capacidad de reacción de los deportistas que probaban el sistema. Luego, la información se centralizaba en una base de datos en la nube. La comunicación entre los distintos componentes del sistema se realizó a través de un protocolo binario propio implementado sobre WiFi (UDP y TCP).}


\section{Proyectos}
\cventry{2020}{GeminiForm}{Web}{React}{}{Desarrollo personal de un software que identifica los principales problemas a la hora de trabajar con formularios en React y brinda una API limpia, amigable y simple de usar. Entre las principales características que ofrece esta librería es debouncing de manera built in y validaciones asincrónicas (junto la cancelación de las mismas). Para el desarrollo de la misma, fue necesario contar con los conocimientos más avanzados en React y programación funcional (Hooks). El proyecto está en fase beta pero ya fue presentado a varias personas de la comunidad de React para recibir feedback.}

\cventry{2020}{Ingresos}{Web}{React}{}{Desarrollo personal de un software que permite registrar tus movimientos de entrada y salida, agrupándolos por categorías. La información se guarda en Google Firebase y luego puede consultarse de diferentes maneras gracias a una serie de filtros proporcionados.
\href{https://github.com/Agusss19/react-ingresos}{https://github.com/Agusss19/react-ingresos}}

\cventry{2018}{Spotlight}{Microcontroladores}{Golang}{}{Desarrollo de un software que emula el comportamiento de un microcontrolador capaz de recibir, vía WiFi, determinados comandos a través de un protocolo binario y emitir por consola los diferentes valores de salidas. Tanto el emulador como el protocolo fue escrito en el lenguaje Golang. \href{https://github.com/Agusss19/go-qsy-terminal}{https://github.com/Agusss19/go-qsy-terminal}}

\cventry{2017}{Cartel Luminoso}{Microcontroladores}{C++}{}{Desarrollo de un panel de LEDs controlado a través de WiFI por un microcontrolador ESP8266. Se requirió conocimientos de C++, protocolos de comunicación de redes, TLS, certificados, concurrencia y electrónica. \href{https://github.com/TPI-2017/esp}{https://github.com/TPI-2017/esp}}

\cventry{2017}{Semáforos Inteligentes}{Microcontroladores}{C++}{}{Creación de un sistema concurrente de semáforos inteligentes que permite el control de congestión del tráfico en función de la presencia de autos que exista en cada rama de una intersección. El trabajo fue realizado con Arduino y la tecnología FreeRTOS. \href{https://github.com/real-time-unlp/traffic-light-controller}{https://github.com/real-time-unlp/traffic-light-controller}}

\section{Idiomas}
\cvitemwithcomment{Nativo}{Español}{}
\cvitemwithcomment{Intermedio}{Ingles}{}

\section{Conocimientos técnicos}
\cvdoubleitem
{
	\cvlistitem{.NET (C\#)}
	\cvlistitem{Git}
	\cvlistitem{Java}
	\cvlistitem{Golang}
}
{
}
{
	\cvlistitem{React}
	\cvlistitem{Angular}
	\cvlistitem{Javascript}
	\cvlistitem{Typescript}
}
{
    \cvlistitem{Redux}
	\cvlistitem{Latex}
	\cvlistitem{C++}
	\cvlistitem{Android}
}
\section{Aplicaciones laborales}
\cvdoubleitem
{
	\cvlistitem{Code}
	\cvlistitem{VS}
	\cvlistitem{IntelliJ}
	\cvlistitem{Eclipse}
}
{
}
{
    \cvlistitem{Jira}
	\cvlistitem{Jenkins}
	\cvlistitem{Slack} 
	\cvlistitem{Trello}
}
{
	\cvlistitem{AzureDataStudio}
	\cvlistitem{SQLManagment}
	\cvlistitem{React Dev Tools}
	\cvlistitem{OracleSQL} 
}

\section{Asistencias y eventos}
\cvlistitem{Participación en El Simposio Argentino de Sistemas Embebidos (SASE) 2017- Tutoriales.}
\cvlistitem{Mención especial como uno de los mejores promedios del ciclo lectivo 2017-2018 en los premios Joaquín V. Gonzalez llevado a cabo por la Municipalidad de La Plata.}

\end{document}


%% end of file `template.tex'.
